
\documentclass[12pt,a4paper]{article}
\usepackage{titlesec} %these are how we import packages, one helps set up footers and title layout
\usepackage{fancyhdr}
\usepackage{titlesec}
\newcommand{\sectionbreak}{\clearpage}
\usepackage{apacite}
% !TEX TS-program = pdflatex
% !TEX encoding = UTF-8 Unicode
\usepackage[utf8]{inputenc} % set input encoding (not needed with XeLaTeX)
\usepackage{graphicx} % support the \includegraphics command and options

% \usepackage[parfill]{parskip} % Activate to begin paragraphs with an empty line rather than an indent

%%% PACKAGES
\usepackage{booktabs} % for much better looking tables
\usepackage{array} % for better arrays (eg matrices) in maths
\usepackage{paralist} % very flexible & customisable lists (eg. enumerate/itemize, etc.)
\usepackage{verbatim} % adds environment for commenting out blocks of text & for better verbatim
\usepackage{subfig} % make it possible to include more than one captioned figure/table in a single float
\usepackage[toc,page]{appendix}
% These packages are all incorporated in the memoir class to one degree or another...
\usepackage{url}

%header and footer settings
\pagestyle{fancyplain}
\fancyhf{}
\renewcommand{\headrulewidth}{0.5pt}
\renewcommand{\footrulewidth}{0.5pt}
\setlength{\headheight}{15pt}
\fancyhead[L]{Jacob Barrow - 40337360}
\fancyhead[R]{ SOC10101 Honors Project}
\fancyfoot[L]{}
\fancyfoot[C]{\thepage}

%this starts the document
\begin{document}

\newcommand{\HRule}{\rule{\linewidth}{0.5mm}}

\begin{titlepage}
	\begin{center}

	\HRule \\[0.4cm]
    	{\Large \bfseries Detecting Incongruous News Articles \par}
	\vspace{0.2cm}
	\HRule \\[1.5cm]

	
    	\vspace{3cm}
	\begin{minipage}{0.4\textwidth}
	\begin{center} \large
        \emph{}\\
        	Jacob Barrow - 40337360
				
   	 \end{center}
    	\end{minipage}
	
	\vspace{2cm}
    	\begin{minipage}{1\textwidth}
    	\begin{center} \large
        
		Submitted in partial fulfilment of\\
		the requirements of Edinburgh Napier University\\
		for the Degree of\\
        	BSc (Hons) Computing Science
    	\end{center}
    	\end{minipage}

    	\vfill

    	% Bottom of the page
	\begin{minipage}{1\textwidth}
    	\begin{center} \large
		School of Computing
    	\end{center}
    	\end{minipage}
	
	\vspace{1cm}
    	{\large \today}


	\end{center}
\end{titlepage}
%{\large Submitted in partial fulfilment of the requirements of Edinburgh Napier University for the Degree of }

\section*{Authorship Declaration}
\vspace{0.5cm}
\begin{flushleft}
I, (Insert Name eg. Norman Stanley Fletcher), confirm that this dissertation and the work presented in it are my own achievement.\newline

Where I have consulted the published work of others this is always clearly attributed;\newline

Where I have quoted from the work of others the source is always given. With the exception of such quotations this dissertation is entirely my own work;\newline

I have acknowledged all main sources of help; \newline

If my research follows on from previous work or is part of a larger collaborative research project I have made clear exactly what was done by others and what I have contributed myself;\newline

I have read and understand the penalties associated with Academic Misconduct.\newline

I also confirm that I have obtained informed consent from all people I have involved in the work in this dissertation following the School's ethical guidelines.\newline
\end{flushleft}

\begin{flushleft} \large
\emph{Signed:} \\
\end{flushleft}

\vspace{.5cm}

\begin{flushleft} \large
\emph{Date:} \\
\end{flushleft}

\vspace{.5cm}

\begin{flushleft} \large
\emph{Matriculation no: }  \\
\end{flushleft}
\pagebreak

\section*{General Data Protection Regulation Declaration}
\vspace{0.5cm}
\begin{flushleft}
Under the General Data Protection Regulation (GDPR) (EU) 2016/679, the University cannot disclose your grade to an unauthorised person. However, other students benefit from studying dissertations that have their grades attached. \newline

\vspace{0.5cm}

Please sign your name below one of the options below to state your preference.\newline
\vspace{0.5cm}

The University may make this dissertation, with indicative grade, available to others.\newline
\vspace{3cm}


The University may make this dissertation available to others, but the grade may not be disclosed.\newline
\vspace{3cm}


The University may not make this dissertation available to others.\newline
\end{flushleft}



\pagebreak

\begin{abstract}
% fill the abstract in here
\end{abstract}
\pagebreak

\tableofcontents % is generated for you
\newpage

\listoftables
\newpage

\listoffigures
%you may have captions such as equations, listings etc they should all appear as required
%these are done for you as long as you use \begin{figure}[placement settings] .. bla bla ... \end{figure}
\newpage

\section*{Acknowledgements}
Insert acknowledgements here
\subsection*{}
	I would like to thank...
\newpage


\section{Introduction}
You can fill out sections as you please.


\subsection{Overview Of Project Milestones}

This is a sub sub section with a list of bullet points.
\begin{itemize}\itemsep0pt
	\item A working X, that will be used for this investigation.
	\item Investigation of current tools and their potential use during an investigation of X .
	\item Programming of X with related frameworks Y and Z.
	\item That is all.
\end{itemize}


\section{Background Research}

\subsection{Existing Approaches}
A range of studies have already considered several aspects of this project.

\citeA{manjesh2017} used Natural Language Processing (NLP) to identify 'clickbait' headlines, however they did not consider the headline's relationship to the article.

\citeA{}
 


\subsection{Possible Approaches}
\subsubsection{Neural Networks and Deep Learning}
\subsubsection{Natural Language Processing}

\section{Data Collection}

\subsection{Attributes}
Before collecting the data, it's important to decide what form it'll take and what attributes will be stored.

As the aim of the project is to identify incongruence between an articles headline and body, these two attributes will be included in the dataset. In order to identify trends and allow for further analysis, the article's date of publication and the publisher (e.g. BBC, The Guardian, etc.) will also be stored.

Collection could have gone further and retained the articles category (e.g. 'politics', 'sport' etc.), but different publishers categorise articles in different ways - for instance, the BBC has a combined 'Science and Environment' category, whereas The Guardian splits these into two distinct categories. Additionally, similar news articles can be filed under different categories, depending on the publisher. As this project's focus is on the article's content, and not categorisation, it can be considered out of scope to investigate the interplay between different publisher's approach to categorising articles.

\subsection{Sources}
The Independent is one of the only online publishers to make available their entire archives. Using the methods mentioned in Section \ref{obtaining-data}, XXX articles were collected, from 2011 to 2020. This 9-year period should prove a useful dataset to analyse a potentially changing landscape in the congruity of news headlines.

The BBC has an 'On This Day' page\footnote{\url{http://news.bbc.co.uk/onthisday}} that has a very select archive from 1950-2005, and analysing these articles could produce some interesting results. However, each of these articles will have been hand-picked (as evidenced by the 'In Context' notes alongside each article), and only represent historic world news events. Therefore, these articles will not be a suitable representation for the average of the time period they are from.

As well as archives, current news was also collected from a range of publishers. A varied range of UK publishers were selected, in order to create a cross

Table \ref{tab:data-sources} shows the full list of data sources collected, as well as the time range they cover and the total records obtained.

\begin{table}[]
\begin{tabular}{llll}
\textbf{Publisher} & \textbf{Earliest} & \textbf{Latest} & \textbf{Total} \\
BBC (Archive) & 1950-01-21 & 2005-12-11 & 1857 \\
The Independent (Archive) & 2011-01-01 & XXX & XXX \\
BBC (Daily) & 2019-04-18 & XXX & XXX \\
Daily Mail (Daily) & 2019-09-24 & XXX & XXX \\
The Guardian (Daily) & 2020-06-25 & XXX & XXX \\
\end{tabular}
\caption{Extents of the data sources collected}
\label{tab:data-sources}
\end{table}

\subsection{Obtaining the Data} \label{obtaining-data}
To collect the data, several Python scripts were created. For the daily news, the publishers' various RSS feeds were consulted, and for the archives a more customised approach was taken.

These scripts utilise the BeautifulSoup library to parse each article's webpage and scrape them for the headline, date and body text. As each publisher builds their websites using different design patterns and with different technologies, each script had to be tailor made to fit the page structure.

In addition, some sites implemented a strict rate-limit on requests - to make a copy of The Independent's archive took around XXX days to complete, scraping one article every 15 seconds. 

\subsection{Ethics}
Across a variety of datasets, XXX articles were collected for analysis. This is a substantial amount of data, and represents the work of many individual journalists and news publishers. 

While automated techniques were used to collect the data, everything collected was publically accessiable. In addition, it is legal to make a digital copy of copyrighted data for non-commercial research \cite{web_govcopyright}. Even so, care still needs to be taken in the obtainment of the data in order to avoid overloading or altering the regular service of these archives. As mentioned above, requests were rate-limited to avoid inadvertant denial of service attack, and spread out over a long period of time. Additionally, the rolling news was only collected once per day, at midnight, in order to minimise the impact of the scraping.


\bibliographystyle{apacite}
\bibliography{bibliography}

%you can crate this on a extra tex document just like the title or any other part of the document.
\newpage
\begin{appendices}
\section{Project Overview}
%insert IPO

\begin{subappendices}
\subsection{Example sub appendices}
...
\end{subappendices}

\section{Second Formal Review Output}
Insert a copy of the project review form you were given at the end of the review by the second marker

\section{Diary Sheets (or other project management evidence)}
Insert diary sheets here together with any project management plan you have

\section{Appendix 4 and following}
insert content here and for each of the other appendices, the title may be just on a page by itself, the pages of the appendices are not numbered, unless an included document such as a user manual or design document is itself pager numbered.
\end{appendices}

\end{document}
