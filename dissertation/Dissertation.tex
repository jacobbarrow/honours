
\documentclass[12pt,a4paper]{article}
\usepackage{titlesec} %these are how we import packages, one helps set up footers and title layout
\usepackage{fancyhdr}
\usepackage{titlesec}
\newcommand{\sectionbreak}{\clearpage}
\usepackage{apacite}
% !TEX TS-program = pdflatex
% !TEX encoding = UTF-8 Unicode
\usepackage[utf8]{inputenc} % set input encoding (not needed with XeLaTeX)
\usepackage{graphicx} % support the \includegraphics command and options

% \usepackage[parfill]{parskip} % Activate to begin paragraphs with an empty line rather than an indent

%%% PACKAGES
\usepackage{booktabs} % for much better looking tables
\usepackage{array} % for better arrays (eg matrices) in maths
\usepackage{paralist} % very flexible & customisable lists (eg. enumerate/itemize, etc.)
\usepackage{verbatim} % adds environment for commenting out blocks of text & for better verbatim
\usepackage{subfig} % make it possible to include more than one captioned figure/table in a single float
\usepackage[toc,page]{appendix}
% These packages are all incorporated in the memoir class to one degree or another...
\usepackage{url}

%header and footer settings
\pagestyle{fancyplain}
\fancyhf{}
\renewcommand{\headrulewidth}{0.5pt}
\renewcommand{\footrulewidth}{0.5pt}
\setlength{\headheight}{15pt}
\fancyhead[L]{Joe Bloggs - 2201882}
\fancyhead[R]{ SOC10101 Honors Project}
\fancyfoot[L]{}
\fancyfoot[C]{\thepage}

%this starts the document
\begin{document}

%you can import other documents into your main one, these layout the Title and Declarations on its own page.
%you might need to change these to \ if your on Microsoft Windows.
\input{./Dissertation-Title.tex}
\input{./Dissertation-Dec.tex}
\pagebreak
\input{./Dissertation-DP.tex}
\pagebreak

%LaTeX let you define the abstract separately so it wont get sucked into the main document.
\begin{abstract}
% fill the abstract in here
\end{abstract}
\pagebreak

\tableofcontents % is generated for you
\newpage

\listoftables
%generated in same way as figures
\newpage

\listoffigures
%you may have captions such as equations, listings etc they should all appear as required
%these are done for you as long as you use \begin{figure}[placement settings] .. bla bla ... \end{figure}
\newpage

\section*{Acknowledgements}
Insert acknowledgements here
\subsection*{}
	I would like to thank...
\newpage


\section{Introduction}
You can fill out sections as you please.


\subsection{Overview Of Project Milestones}

This is a sub sub section with a list of bullet points.
\begin{itemize}\itemsep0pt
	\item A working X, that will be used for this investigation.
	\item Investigation of current tools and their potential use during an investigation of X .
	\item Programming of X with related frameworks Y and Z.
	\item That is all.
\end{itemize}


\section{Literature Review}
The following bibliographic...

% another example section
\section{Data}
In order to perform classification, the algorithm will need a dataset to classify. 

\subsection{Data Attributes}
Before collecting the data, it's important to decide what form it'll take and what attributes will be stored.

As the aim of the project is to identify incongruence between an articles headline and body, these two attributes will be included in the dataset. In order to identify trends and allow for further analysis, the article's date of publication and the publisher (e.g. BBC, The Guardian, etc.) will also be stored.

Collection could have gone further and retained the articles category (e.g. 'politics', 'sport' etc.), but different publishers categorise articles in different ways - for instance, the BBC has a combined 'Science and Environment' category, whereas The Guardian splits these into two distinct categories. Additionally, similar news articles can be filed under different categories, depending on the publisher. As this project's focus is on the article's content, and not categorisation, it can be considered out of scope to investigate the interplay between different publisher's approach to categorising articles.

\subsection{Data Timeliness}
An interesting trend to look at will be how, if at all, incongruity in articles has changed over time. However, most mainstream news sites don't have an accesiable way of listing old news. 

The BBC has an 'On This Day' page \cite{web_bbconthisday} that has a very select archive from 1950-2005, and analysing these articles could produce some interesting results. However, each of these articles will have been hand-picked (as evidenced by the 'In Context' notes alongside each article), and only represent historic world news events. Therefore, these articles will not be a suitable representation for the average of the time period they are from.

\subsection{Data Collection}


\bibliographystyle{apacite}
\bibliography{bibliography}

%you can crate this on a extra tex document just like the title or any other part of the document.
\newpage
\begin{appendices}
\section{Project Overview}
%insert IPO

\begin{subappendices}
\subsection{Example sub appendices}
...
\end{subappendices}

\section{Second Formal Review Output}
Insert a copy of the project review form you were given at the end of the review by the second marker

\section{Diary Sheets (or other project management evidence)}
Insert diary sheets here together with any project management plan you have

\section{Appendix 4 and following}
insert content here and for each of the other appendices, the title may be just on a page by itself, the pages of the appendices are not numbered, unless an included document such as a user manual or design document is itself pager numbered.
\end{appendices}

\end{document}
