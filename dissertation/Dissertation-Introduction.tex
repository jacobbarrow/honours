\section{Introduction}
Fake news, disinformation and misleading articles have been the subject of much debate and controversy. With the decline of traditional print media, more people turn to online sources for news, where publishing has a much lower barrier to entry.

Incongruous headlines misrepresent the content of an article and, intentional or not, can lead to the formation of unfounded opinions and misconstrued facts. This dissertation seeks to create a classifier capable of calculating a measure of a headline's incongruence, by utilising Natural Language Processing (NLP) to analyze the components of a news article.

\subsection{Aims}
The primary aim of this dissertation is to create an incongruence classifier. Additionally, a large dataset of news articles will be created and a subsection of them labelled.


\subsection{Project Outline}
The project will consist of a range of deliverables and milestones.
 
\paragraph{Background Research}
A literature review will be undertaken to provide context to the project and define its scope

\paragraph{Data collection}
In order to provide the algorithm with articles to classify, a dataset is required. It will be sourced from a range of publishers, and ideally cover a large timescale to allow trends to be identified.

\paragraph{Data Labelling}
A raw dataset can be made more valuable by creating a training set from it. This will be a small subsection of the articles that volunteers will label with their perception of the incongruence.

\paragraph{Algorithm Creation}
Once a dataset is generated and labelled, the algorithm to classify the articles can be implemented. 

\paragraph{Analysis}
The algorithm will create a range of data points that can be analysed. Trends will be spotted in the dataset used, and comparisons made with different implementations of the algorithm.
