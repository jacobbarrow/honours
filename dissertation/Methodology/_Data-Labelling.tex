\subsection{Data Labelling}
\citeA{chesney2017} reviews the current datasets available to detect incongruent articles and concludes that while many are available and have some potential use, none are a good fit for the task. With a lack of a well suited labelled dataset, a bespoke one was created.

\subsubsection{Generating a Dataset}
To create a subset to be labelled, 150 articles were randomly selected from the collected articles' primary dataset. As this subset was taken before collection had been completed and the dataset finalised, the articles selected will not represent the most recent in the dataset. However, as the time frame excluded (around two months) is relatively insubstantial, this should have a trivial effect on the data quality.

\subsubsection{Design and Implementation of Labelling Site}
The site used to collect ratings for articles was bespoke - existing off-the-shelf survey solutions did not meet this approach's needs or would have to be heavily modified to suit the data.

The labelling site's design was kept minimalist and clean, ensuring that the volunteers would not find themselves hindered and could concentrate on the task at hand. 

Two pages were created, one to show the consent and briefing information, the other to show an article and allow a user to select a rating. Appendix \ref{app:rating-form} shows the form controls used to rate each article, and a full view of both the page's contents can be seen on the project's GitHub repository \footnote{\url{https://github.com/jacobbarrow/honours/tree/master/data-labelling/views}}.

As well as the rating (codified as an integer from 0-6), a javascript file measured the time someone took to read and rate the article. This will allow for a more extensive analysis of the data and aid in compiling the final labelled set.

The site used Flask, a Python framework, to serve the pages and interact with the SQLite database.

\subsubsection{Ethics}
 While no personal information is collected, and it is impossible to identify an individual from the data, the site asks volunteers to participate in research. Therefore, ethical approval must be acquired.

Appendix \ref{app:ethics-participant-info} shows the participant information sheet shown to all participants before they proceeded to the rating. 

Appendix \ref{app:ethics-approval} is the ethical approval form that details the extent of the information collected and how it will be processed and retained.

\subsubsection{Analysis of Labelled Data}
The site was distributed through several channels (friends, online forums), and 159 ratings were received. Unfortunately, this represents roughly one rating per article, which is not enough to form an accurate picture of incongruence - multiple ratings would be required so an average could be taken. This means the labelled dataset cannot be used in the training and analysis of a classifier.