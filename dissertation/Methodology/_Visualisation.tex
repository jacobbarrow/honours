\subsection{Visualisation}

Creating a platform to visualise and identify trends using the NLP approaches discusses and used in section \ref{experimentation} will allow for ease of analysis, demonstrate areas for further work, and aid other researchers with the management of the dataset.

In order to be effective, the visualiser should:
\begin{itemize}
	\item Allow for a downloadable customised export of the data to CSV format
	\item Be able to generate graphs, using a range of NLP approaches to create datapoints based on the dataset
	\item Have a non-obstrusive, simple design	
\end{itemize}


\subsubsection{Design and Frontend Implementation}
To keep things simple, the design of the visualiser was kept to a one-page layout. Appendix \ref{app:vis-wireframe} shows a rudimentary wireframe of this layout. 

The front end was written in HTML and CSS, with accessiability as a key concern. Semantic HTML5 elements were used, which enhances screen reader navigation and comprehension, and the form has a logical structure. Additionally, changes to the dynamic element of the form (changing the analysis technique used) was broadcast using the ARIA standard\footnote{\url{https://developer.mozilla.org/en-US/docs/Web/Accessibility/ARIA/ARIA_Live_Regions}}.

\subsubsection{Implementation}
Here's the implementation deets

\subsubsection{Analysis}
Here's how well it works
